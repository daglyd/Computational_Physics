
\documentclass[english,notitlepage]{revtex4-1}  % defines the basic parameters of the document
%For preview: skriv i terminal: latexmk -pdf -pvc filnavn



% if you want a single-column, remove reprint

% allows special characters (including æøå)
\usepackage[utf8]{inputenc}
%\usepackage[english]{babel}

%% note that you may need to download some of these packages manually, it depends on your setup.
%% I recommend downloading TeXMaker, because it includes a large library of the most common packages.

\usepackage{physics,amssymb}  % mathematical symbols (physics imports amsmath)
\include{amsmath}
\usepackage{graphicx}         % include graphics such as plots
\usepackage{xcolor}           % set colors
\usepackage{hyperref}         % automagic cross-referencing (this is GODLIKE)
\usepackage{listings}         % display code
\usepackage{subfigure}        % imports a lot of cool and useful figure commands
\usepackage{float}
%\usepackage[section]{placeins}
\usepackage{algorithm}
\usepackage[noend]{algpseudocode}
\usepackage{subfigure}
\usepackage{tikz}
\usetikzlibrary{quantikz}
% defines the color of hyperref objects
% Blending two colors:  blue!80!black  =  80% blue and 20% black
\hypersetup{ % this is just my personal choice, feel free to change things
    colorlinks,
    linkcolor={red!50!black},
    citecolor={blue!50!black},
    urlcolor={blue!80!black}}

%% Defines the style of the programming listing
%% This is actually my personal template, go ahead and change stuff if you want



%% USEFUL LINKS:
%%
%%   UiO LaTeX guides:        https://www.mn.uio.no/ifi/tjenester/it/hjelp/latex/
%%   mathematics:             https://en.wikibooks.org/wiki/LaTeX/Mathematics

%%   PHYSICS !                https://mirror.hmc.edu/ctan/macros/latex/contrib/physics/physics.pdf

%%   the basics of Tikz:       https://en.wikibooks.org/wiki/LaTeX/PGF/Tikz
%%   all the colors!:          https://en.wikibooks.org/wiki/LaTeX/Colors
%%   how to draw tables:       https://en.wikibooks.org/wiki/LaTeX/Tables
%%   code listing styles:      https://en.wikibooks.org/wiki/LaTeX/Source_Code_Listings
%%   \includegraphics          https://en.wikibooks.org/wiki/LaTeX/Importing_Graphics
%%   learn more about figures  https://en.wikibooks.org/wiki/LaTeX/Floats,_Figures_and_Captions
%%   automagic bibliography:   https://en.wikibooks.org/wiki/LaTeX/Bibliography_Management  (this one is kinda difficult the first time)
%%   REVTeX Guide:             http://www.physics.csbsju.edu/370/papers/Journal_Style_Manuals/auguide4-1.pdf
%%
%%   (this document is of class "revtex4-1", the REVTeX Guide explains how the class works)


%% CREATING THE .pdf FILE USING LINUX IN THE TERMINAL
%%
%% [terminal]$ pdflatex template.tex
%%
%% Run the command twice, always.
%% If you want to use \footnote, you need to run these commands (IN THIS SPECIFIC ORDER)
%%
%% [terminal]$ pdflatex template.tex
%% [terminal]$ bibtex template
%% [terminal]$ pdflatex template.tex
%% [terminal]$ pdflatex template.tex
%%
%% Don't ask me why, I don't know.

\begin{document}

\title{Project 1 - FYS3150}      % self-explanatory
\author{Dag Arne Lydvo}          % self-explanatory
\date{\today}                             % self-explanatory
\noaffiliation                            % ignore this, but keep it.


\maketitle 
    
\textit{Github repo: $https://github.com/daglyd/Computational_Physics$}
    
\section*{Problem 1}

\[-\frac{d^2u}{dx^2}=f(x)=100e^{-10x} \]
Solving for u:
\[-u = \int\int 100e^{-10x} dx^2\]
Using $w = -10x$ and $dx=\frac{dw}{-10}$
\[-u = 100\int\int e^{w}\frac{dw^2}{-10}\]
\[ -u = -10 \int e^{-10x}+C_1 dx\]
\[-10\int e^w +C_1 \frac{dw}{-10} = e^w + C_1w + C_2\]
\[-u = e^{-10x} +C_1(-10x)+C_2\]
Solving for constants using boundary conditions $u(0)=u(1)=0$
\[-u(0)=0=e^0 +C_2 \rightarrow C_2=-1\]
\[-u(1)=0=e^{-10}+C_1(-10)-1 \rightarrow C_1=\frac{1}{10}(e^{-10}-1)\]
\[-u(x) = e^{-10x}+\frac{1}{10}(e^{-10}-1)(-10x)-1\]
\[u(x)=1-(1-e^{-10})x - e^{-10x}\]


\section*{Problem 2}
\begin{figure}[H]
	\centering
	\includegraphics[scale=0.75]{../Figures/problem_2.pdf}
	\caption{Exact solution of u(x) over 1000 steps of x.}
\end{figure}




\section{Problem 3}
\[-\frac{d^2u}{dx^2}=f(x)\]

The continous variable x becomes discrete, $x\rightarrow x_1$\\
Where i = (0,1,...,n), where n is the number of steps. 

$u(x)$ becomes $u(x_i)$, and $u(x_1)=u(x_0+h)$,
where h is the step size defined as $ h = \frac{x_n-x_0}{n}$

The discrete version of $-\frac{d^2u}{dx^2}= f(x)$ now becomes \

\[ -\frac{d^2u}{dx^2} \vert_{x_i}= \frac{u_{i+1} + u_{i-1} - 2u_i}{h^2} + O(h^2) \]
This is the discrete version of the double derivative as derived in the lectures,
and the approximation can be written as 
\[v''_i = \frac{-v_{i+1}-v_{i-1}+2v_i}{h^2} = f_i\]


\section{Problem 4}

The approximte discrete version of the Poisson equation from problem 3 can be rewritten as \[-v_{i+1}-v_{i-1}+2v_i = f_i h^2\]

Here we have n numbers of unknown variables $v_i$ for $i=(0,1,...,n)$ making up the the vector $\vec{v}=(v_0,v_1,...,v_n)$. We also have n numbers of known functions $g_i = f_i h^2$ making up the vector $\vec{g}=(f_0h^2,f_1h^2,...,f_nh^2)$. 

The termns of $v'' h^2 = v_{i-1}-2v_i+v_{i+1}$ makes up  n sets of equations. 

\[2v_1- v_2 = g_1\]
\[-v_1 + 2v_2 - v_3 = g_2\]
 \[ .  .  . \]
  \[.  .  . \]
  \[.  .  . \]
\[-v_{n-1} + 2v_n = g_n\]

The terms of $-v_0 and -v_{n+1}$ are zero because of the boundary conditions. The first and last equations therefore contains only two terms. 

This set of equations can then be written as the matrix equation $\vb{A}\vec{v} = \vec{g}$. \\
Where $\vb{A}$ is:
\[
\begin{bmatrix}
 2  & -1 & 0 &\dots & 0 \\
 -1 & 2  & -1 & \dots & 0 \\
 \vdots & \vdots & \ddots &\dots & 0 \\
 0 & 0 &-1 & 2 & -1 \\
 0 & 0 & 0& -1 & 2 & 
\end{bmatrix} \]\\
and $\vec{v}$ and $ \vec{g}$ is: 
\[\begin{bmatrix}
	v_1 \\ v_2 \\ \vdots \\ v_n
\end{bmatrix}\begin{bmatrix}
	g_1 \\ g_2 \\ \vdots \\ g_n
\end{bmatrix}\]


\section{Problem 5}
Was a bit confused about this one, and have not got the time to look at it closely. \\ 
Maybe the boundary conditions are not included in the solutions $\vec{v}$ but are in $\vec{v}^*$? 
\section{Problem 6}
Solving a general tridiagonal matrix can be done by two general steps. First forward substitution then backward substitution on the subdiagonal $a_i$, main diagonal $b_i$ and superdiagonal $c_i$.  
\begin{algorithm}[H]
	\caption{Solving general tridiagonal matrix}\label{algo:midpoint_rule}
	\begin{algorithmic}
		\State Forward substitution
		\State $\tilde{b_0}=b_0$  \Comment{Here's a comment}
		\State $\tilde{g_0}=g_0$
		\For{$i =  1, ..., n$}
		\State $\tilde{b_i} = b_i - \frac{a_i}{\tilde{b_{i-1}}}c_{i-1}$ \Comment{3(n-1) FLOPs}
		\State $\tilde{g_i} = g_i - \frac{a_i}{\tilde{b_{i-1}}}\tilde{g_{i-1}}$ \Comment{3(n-1) FLOPs}
		\EndFor
		
		\State Backward substitution 
		\State $v_n = \frac{\tilde{g_n}}{\tilde{b_n}}$ \Comment{1 FLOP}
		\For{$i = n-1,n-2, ...,1}$
		\State $v_i = \frac{\tilde{g_i}-c_iv_{i+1}}{\tilde{b_i}}$ \Comment{3(n-1) FLOPs}
		\EndFor
		
	\end{algorithmic}
\end{algorithm}

With the forward subs. not counting the first element $i=0$ and the backward substitution not counting the last element $i=n$ both for loops runs for $n-1$ loops. 
The total FLOPs for the algorithm then becomes $9(n-1) + 1$


\section{Problem 7}
The numerical solutions where $n=10$ seems to be really bad. As n increases the plot show a closer alignment to the exact solution. And at $n=1000$ it seems to be very close by eye. 
\begin{figure}[H]
	\centering
	\includegraphics[scale=0.5]{../Figures/problem7.pdf}
	\caption{Numerical solution using the general algorithm with varying values of n vs the exact solution u(x)}
\end{figure}
\newpage
\section{Problem 8}
The numerical solutions seems to improve as n increases, and at a even rate. 
\begin{figure}[H]
	\centering
	\includegraphics[scale=0.5]{../Figures/problem8_abs_error.pdf}
	\caption{The $log_{10}$ of the absolute error for differing values of n.}
\end{figure}
\begin{figure}[H]
	\centering
	\includegraphics[scale=0.5]{../Figures/problem8_rel_error.pdf}
	\caption{The $log_{10}$ of the relative error for differing values of n.}
\end{figure}
In the table below I have extracted max values for the relative error for different values of n. At x=0 errors became infinite, while blowing up at x=1. I have therefore included columns where inf have been excluded as well as x=1. 
The relative error seems to be approaching 1, indicating a more accurate numerical solutions as n increases. 
\begin{figure}[H]
	\centering
	\includegraphics[scale=1]{../Figures/problem8_table.pdf}
	\caption{The max relative error for differing values of n. }
\end{figure}



\section{Problem 9}

For the special case where the matrix $\vb{A}$ as a signature of (-1,2,-1), the diagonal vectors have constant values the general algorithm will be modified to possible reduce the number of FLOPs.
 
Since the values for $a_i$ and $c_i$ are constant the operations between these values can be removed. And since the value of $a$ and $c$ are $-1$ we can just switch the operator.
This special algorithm reduces the number of FLOPs from $9(n-1)+1$ to $6(n-1)+1$. 
\begin{algorithm}[H]
	\caption{Solving special tridiagonal matrix}\label{algo:midpoint_rule}
	\begin{algorithmic}
		\State Forward substitution
		\State $\tilde{b_0}=b_0$  \Comment{Here's a comment}
		\State $\tilde{g_0}=g_0$
		\For{$i =  1, ..., n$}
		\State $\tilde{b_i} = b_i - \frac{1}{\tilde{b_{i-1}}}$ \Comment{2(n-1) FLOPs}
		\State $\tilde{g_i} = g_i + \frac{\tilde{g_{i-1}}}{\tilde{b_{i-1}}}$ \Comment{2(n-1) FLOPs}
		\EndFor
		
		\State Backward substitution 
		\State $v_n = \frac{\tilde{g_n}}{\tilde{b_n}}$ \Comment{1 FLOP}
		\For{$i = n-1,n-2, ...,1}$
		\State $v_i = \frac{\tilde{g_i}+v_{i+1}}{\tilde{b_i}}$ \Comment{2(n-1) FLOPs}
		\EndFor
		
	\end{algorithmic}
\end{algorithm}


\section{Problem 10}
Both the general and the special algorithm have been run 10 times and the mean time have then been calculated. \\
The steps used have started at 10 and increased by a factor of 10 to $10^6$. \\
As n increases the gain in time saved for the special algorithm are increasing, suggesting a faster algorithm than the general case. \\
The seemingly linear increase in time for both the general and special algorithms is consistent with the linear nature of number of FLOPs for the method.  
\begin{figure}[H]
	\centering
	\includegraphics[scale=0.5]{../Figures/problem10.pdf}
	\caption{Numerical solution using the general algorithm with varying values of n vs the exact solution u(x)}
\end{figure}

\section{Problem 11}
Time is running out for me now, and I have not been able to do this problem. 

\end{document}
